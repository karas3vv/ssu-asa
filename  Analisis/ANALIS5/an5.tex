\documentclass[otchet]{SCWorks}
% Тип обучения (одно из значений):
%    bachelor   - бакалавриат (по умолчанию)
%    spec       - специальность
%    master     - магистратура
% Форма обучения (одно из значений):
%    och        - очное (по умолчанию)
%    zaoch      - заочное
% Тип работы (одно из значений):
%    coursework - курсовая работа (по умолчанию)
%    referat    - реферат
%  * otchet     - универсальный отчет
%  * nirjournal - журнал НИР
%  * digital    - итоговая работа для цифровой кафедры
%    diploma    - дипломная работа
%    pract      - отчет о научно-исследовательской работе
%    autoref    - автореферат выпускной работы
%    assignment - задание на выпускную квалификационную работу
%    review     - отзыв руководителя
%    critique   - рецензия на выпускную работу

% * Добавлены вручную. За вопросами к @mchernigin
\usepackage{preamble}

\begin{document}

% Кафедра (в родительном падеже)
\chair{информатики и программирования}

% Тема работы
\title{Анализ АВЛ"=дерева}

% Курс
\course{2}

% Группа
\group{251}

% Факультет (в родительном падеже) (по умолчанию "факультета КНиИТ")
% \department{факультета КНиИТ}

% Специальность/направление код - наименование
% \napravlenie{02.03.02 "--- Фундаментальная информатика и информационные технологии}
% \napravlenie{02.03.01 "--- Математическое обеспечение и администрирование информационных систем}
% \napravlenie{09.03.01 "--- Информатика и вычислительная техника}
\napravlenie{09.03.04 "--- Программная инженерия}
% \napravlenie{10.05.01 "--- Компьютерная безопасность}

% Для студентки. Для работы студента следующая команда не нужна.
% \studenttitle{студентки}

% Фамилия, имя, отчество в родительном падеже
\author{Карасева Вадима Дмитриевича}

% Заведующий кафедрой
\chtitle{доцент, к.\,ф.-м.\,н.}
\chname{С.\,В.\,Миронов}

% Руководитель ДПП ПП для цифровой кафедры (перекрывает заведующего кафедры)
% \chpretitle{
%     заведующий кафедрой математических основ информатики и олимпиадного\\
%     программирования на базе МАОУ <<Ф"=Т лицей №1>>
% }
% \chtitle{г. Саратов, к.\,ф.-м.\,н., доцент}
% \chname{Кондратова\, Ю.\,Н.}

% Научный руководитель (для реферата преподаватель проверяющий работу)
\satitle{доцент, к.\,ф.-м.\,н.} %должность, степень, звание
\saname{М.\,И.\,Сафрончик}

% Руководитель практики от организации (руководитель для цифровой кафедры)
\patitle{доцент, к.\,ф.-м.\,н.}
\paname{С.\,В.\,Миронов}

% Руководитель НИР
\nirtitle{доцент, к.\,п.\,н.} % степень, звание
\nirname{В.\,А.\,Векслер}

% Семестр (только для практики, для остальных типов работ не используется)
\term{2}

% Наименование практики (только для практики, для остальных типов работ не
% используется)
\practtype{учебная}

% Продолжительность практики (количество недель) (только для практики, для
% остальных типов работ не используется)
\duration{2}

% Даты начала и окончания практики (только для практики, для остальных типов
% работ не используется)
\practStart{01.07.2024}
\practFinish{13.01.2024}

% Год выполнения отчета
\date{2025}

\maketitle

% Включение нумерации рисунков, формул и таблиц по разделам (по умолчанию -
% нумерация сквозная) (допускается оба вида нумерации)
\secNumbering

\tableofcontents

% Раздел "Обозначения и сокращения". Может отсутствовать в работе
% \abbreviations
% \begin{description}
%     \item ... "--- ...
%     \item ... "--- ...
% \end{description}

% Раздел "Определения". Может отсутствовать в работе
% \definitions

% Раздел "Определения, обозначения и сокращения". Может отсутствовать в работе.
% Если присутствует, то заменяет собой разделы "Обозначения и сокращения" и
% "Определения"
% \defabbr

% \intro

% После введения — серии \section, \subsection и т.д.

\section{Пример программы}
\begin{minted}[fontsize=\small, breaklines=true, style=bw, linenos, highlightlines={}]{cpp}
#include <iostream>
#include <algorithm>
using namespace std;

struct Node {
    int key;
    Node* left;
    Node* right;
    int height;

    Node(int k) : key(k), left(nullptr), right(nullptr), height(1) {}
};

int getHeight(Node* node) {
    return node ? node->height : 0;
}

int getBalance(Node* node) {
    return node ? getHeight(node->left) - getHeight(node->right) : 0;
}

void updateHeight(Node* node) {
    node->height = max(getHeight(node->left), getHeight(node->right)) + 1;
}

Node* rotateRight(Node* y) {
    Node* x = y->left;
    Node* T2 = x->right;

    x->right = y;
    y->left = T2;

    updateHeight(y);
    updateHeight(x);

    return x;
}

Node* rotateLeft(Node* x) {
    Node* y = x->right;
    Node* T2 = y->left;

    y->left = x;
    x->right = T2;

    updateHeight(x);
    updateHeight(y);

    return y;
}

Node* balance(Node* node) {
    updateHeight(node);
    int balanceFactor = getBalance(node);

    // Левый перекос
    if (balanceFactor > 1) {
        if (getBalance(node->left) < 0)
            node->left = rotateLeft(node->left);  // LR
        return rotateRight(node);  // LL
    }

    // Правый перекос
    if (balanceFactor < -1) {
        if (getBalance(node->right) > 0)
            node->right = rotateRight(node->right);  // RL
        return rotateLeft(node);  // RR
    }

    return node;
}

Node* insert(Node* node, int key) {
    if (!node)
        return new Node(key);
    if (key < node->key)
        node->left = insert(node->left, key);
    else if (key > node->key)
        node->right = insert(node->right, key);
    else
        return node; // дубликаты не вставляем

    return balance(node);
}

Node* getMinValueNode(Node* node) {
    Node* current = node;
    while (current && current->left)
        current = current->left;
    return current;
}

Node* remove(Node* root, int key) {
    if (!root)
        return root;

    if (key < root->key)
        root->left = remove(root->left, key);
    else if (key > root->key)
        root->right = remove(root->right, key);
    else {
        // Один или ноль потомков
        if (!root->left || !root->right) {
            Node* temp = root->left ? root->left : root->right;
            delete root;
            return temp;
        }

        // Два потомка
        Node* temp = getMinValueNode(root->right);
        root->key = temp->key;
        root->right = remove(root->right, temp->key);
    }

    return balance(root);
}

void inorder(Node* root) {
    if (root) {
        inorder(root->left);
        cout << root->key << " ";
        inorder(root->right);
    }
}

int main() {
    Node* root = nullptr;
    int n, val;

    cout << "Введите количество узлов для вставки: ";
    cin >> n;

    cout << "Введите " << n << " значений:\n";
    for (int i = 0; i < n; ++i) {
        cin >> val;
        root = insert(root, val);
    }

    cout << "Симметричный обход после вставки: ";
    inorder(root);
    cout << endl;

    cout << "Введите значение для удаления: ";
    cin >> val;
    root = remove(root, val);

    cout << "Симметричный обход после удаления: ";
    inorder(root);
    cout << endl;

    return 0;
}
\end{minted}

\section{Вставка в АВЛ"=дереве}
В худшем случае вставка требует времени O(logn), где n-количество элементов в дереве.
Это  происходит из-за необходимости сбалансировать дерево после каждой вставки, что занимает время, пропорциональное высоте дерева.
Следовательно, время вставки для n элементов будет O(nlogn).

\section{Удаление в АВЛ"=дереве}
Как и вставка, удаление также требует времени O(logn) в худшем случае. После удаления элемента дерево также требуется сбалансировать.
Таким образом, время удаления для n элементов также будет O(nlogn).

\section{Поиск в АВЛ"=дереве}
Время поиска в АВЛ-дереве также O(logn) в худшем случае. 
Поиск выполняется по пути от корня до листа в дереве, при этом каждый раз отбрасывается половина оставшихся узлов.

\section{Обходы дерева}
Префиксный (preorder), постфиксный (postorder) и инфиксный (inorder) обходы занимают O(n) времени, так как каждый узел дерева должен быть посещен ровно один раз.
Таким образом, общая временная сложность операций вставки, удаления, поиска и обходов в АВЛ-дереве составляет O(nlogn) для n элементов.


\section{Расход памяти}
\subsection*{Узел дерева}

Для каждого узла дерева выделяется фиксированное количество памяти, состоящее из:
inf: O(1) памяти
left и right: указатели на другие узлы дерева, каждый из которых занимает O(1) памяти
height: переменная типа unsignedchar, занимающая O(1) памяти
Таким образом, общее количество памяти, выделенное под каждый узел дерева, составляет O(1).

\subsection*{Входные данные}
Расход памяти на входные данные (значения, которые вставляются в дерево) также не зависит от размера дерева и может быть считан O(n), где n "--- количество элементов.

\subsection*{Стек вызовов}
Во время выполнения рекурсивных операций используется стек вызовов, чтобы хранить информацию о текущем состоянии выполнения каждой рекурсивной функции. Глубина стека вызовов зависит от высоты дерева и количества рекурсивных вызовов, что может быть O(logn) в худшем случае для операций вставки,удаления и поиска.
Таким образом, общий асимптотический анализ сложности расхода памяти составляет O(n+logn), где n - количество элементов в дереве.

\subsection*{Итог}
Вращение происходит за время O(1). Временная сложность всех функций равна O(logN), потому что дерево всегда сбалансированное.

% Отобразить все источники. Даже те, на которые нет ссылок.
% \nocite{*}

% \begin{thebibliography}{99}
	\bibliographystyle{ugost2003}

% \end{thebibliography}

% Окончание основного документа и начало приложений Каждая последующая секция
% документа будет являться приложением
\appendix

\end{document}
