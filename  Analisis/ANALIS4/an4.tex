\documentclass[otchet]{SCWorks}
% Тип обучения (одно из значений):
%    bachelor   - бакалавриат (по умолчанию)
%    spec       - специальность
%    master     - магистратура
% Форма обучения (одно из значений):
%    och        - очное (по умолчанию)
%    zaoch      - заочное
% Тип работы (одно из значений):
%    coursework - курсовая работа (по умолчанию)
%    referat    - реферат
%  * otchet     - универсальный отчет
%  * nirjournal - журнал НИР
%  * digital    - итоговая работа для цифровой кафедры
%    diploma    - дипломная работа
%    pract      - отчет о научно-исследовательской работе
%    autoref    - автореферат выпускной работы
%    assignment - задание на выпускную квалификационную работу
%    review     - отзыв руководителя
%    critique   - рецензия на выпускную работу

% * Добавлены вручную. За вопросами к @mchernigin
\usepackage{preamble}

\begin{document}

% Кафедра (в родительном падеже)
\chair{информатики и программирования}

% Тема работы
\title{Анализ красно-чёрного дерева}

% Курс
\course{2}

% Группа
\group{251}

% Факультет (в родительном падеже) (по умолчанию "факультета КНиИТ")
% \department{факультета КНиИТ}

% Специальность/направление код - наименование
% \napravlenie{02.03.02 "--- Фундаментальная информатика и информационные технологии}
% \napravlenie{02.03.01 "--- Математическое обеспечение и администрирование информационных систем}
% \napravlenie{09.03.01 "--- Информатика и вычислительная техника}
\napravlenie{09.03.04 "--- Программная инженерия}
% \napravlenie{10.05.01 "--- Компьютерная безопасность}

% Для студентки. Для работы студента следующая команда не нужна.
% \studenttitle{студентки}

% Фамилия, имя, отчество в родительном падеже
\author{Карасева Вадима Дмитриевича}

% Заведующий кафедрой
\chtitle{доцент, к.\,ф.-м.\,н.}
\chname{С.\,В.\,Миронов}

% Руководитель ДПП ПП для цифровой кафедры (перекрывает заведующего кафедры)
% \chpretitle{
%     заведующий кафедрой математических основ информатики и олимпиадного\\
%     программирования на базе МАОУ <<Ф"=Т лицей №1>>
% }
% \chtitle{г. Саратов, к.\,ф.-м.\,н., доцент}
% \chname{Кондратова\, Ю.\,Н.}

% Научный руководитель (для реферата преподаватель проверяющий работу)
\satitle{доцент, к.\,ф.-м.\,н.} %должность, степень, звание
\saname{М.\,И.\,Сафрончик}

% Руководитель практики от организации (руководитель для цифровой кафедры)
\patitle{доцент, к.\,ф.-м.\,н.}
\paname{С.\,В.\,Миронов}

% Руководитель НИР
\nirtitle{доцент, к.\,п.\,н.} % степень, звание
\nirname{В.\,А.\,Векслер}

% Семестр (только для практики, для остальных типов работ не используется)
\term{2}

% Наименование практики (только для практики, для остальных типов работ не
% используется)
\practtype{учебная}

% Продолжительность практики (количество недель) (только для практики, для
% остальных типов работ не используется)
\duration{2}

% Даты начала и окончания практики (только для практики, для остальных типов
% работ не используется)
\practStart{01.07.2024}
\practFinish{13.01.2024}

% Год выполнения отчета
\date{2025}

\maketitle

% Включение нумерации рисунков, формул и таблиц по разделам (по умолчанию -
% нумерация сквозная) (допускается оба вида нумерации)
\secNumbering

\tableofcontents

% Раздел "Обозначения и сокращения". Может отсутствовать в работе
% \abbreviations
% \begin{description}
%     \item ... "--- ...
%     \item ... "--- ...
% \end{description}

% Раздел "Определения". Может отсутствовать в работе
% \definitions

% Раздел "Определения, обозначения и сокращения". Может отсутствовать в работе.
% Если присутствует, то заменяет собой разделы "Обозначения и сокращения" и
% "Определения"
% \defabbr

% \intro

% После введения — серии \section, \subsection и т.д.

\section{Пример программы}
\begin{minted}[fontsize=\small, breaklines=true, style=bw, linenos, highlightlines={}]{cpp}
#include <iostream>
using namespace std;

struct tree // Cтруктура дерева
{
	int inf;
	tree* left;
	tree* right;
	tree* parent;
	bool color; // True - черный, False – красный
};


// Повороты
void left_rotate(tree*& tr, tree* x) // Левый
{
	tree* y = x->right; // Y - правый ребенок X
	x->right = y->left; // Левый ребенок Y становится правым ребенком X  
	if (y->left) // Если левый ребенок Y есть
		y->left->parent = x; // Родителем левого ребенка становится X
	y->parent = x->parent; // Родитель для Y становится родитель X 
	if (x->parent)
	{
		if (x == x->parent->left) // Eсли X левый ребенок 
			x->parent->left = y; // Y становится левым ребенком для родителя X
		else
			x->parent->right = y;// Становится правым ребенком соответственно
	}
	y->left = x; // X становится левым ребенком Y
	x->parent = y; // Y становится родитем X
	if (!y->parent) // Если у Y нет родителя
	{
		tr = y;
		y->color = true; // Y становится корнем и окрашивается в черный цвет
	}
}

void right_rotate(tree*& tr, tree* x) // правый поворот
{
	tree* y = x->left;
	x->left = y->right;
	if (y->right)
		y->right->parent = x;
	y->parent = x->parent;
	if (x->parent)
	{
		if (x == x->parent->left)
			x->parent->left = y;
		else
			x->parent->right = y;
	}
	y->right = x;
	x->parent = y;
	if (!y->parent)
	{
		tr = y;
		y->color = true;
	}
}

// создание узлов
tree* node(tree* p, int x) // красный узел
{
	tree* n = new tree;
	n->inf = x;
	n->parent = p;
	n->right = NULL;
	n->left = NULL;
	n->color = false;
	return n;
}

tree* root(int x) // черный узел
{
	tree* n = new tree;
	n->inf = x;
	n->parent = NULL;
	n->right = NULL;
	n->left = NULL;
	n->color = true;
	return n;
}

tree* grandparent(tree* x) // дедушка
{
	if (x && x->parent)
		return x->parent->parent;
	else
		return NULL;
}

tree* uncle(tree* x) // дядя
{
	tree* g = grandparent(x);
	if (!g)
		return NULL;

	if (x->parent == g->left)
		return g->right;
	else
		return g->left;
}


tree* brother(tree* x) // брат
{
	if (x && x->parent)
	{
		if (x->parent->left == x)
			return x->parent->right;
		else
			return x->parent->left;
	}
	else
		return NULL;
}

void preorder_color(tree* tr) // прямой обход с цветами (К-Л-П)
{
	if (tr)
	{
		cout << tr->inf; // корень
		if (tr->color == true)
			cout << " - b  ";
		else
			cout << " - r  ";
		preorder_color(tr->left); // левый 
		preorder_color(tr->right); // правый 
	}
}

// вставка
void insert_case5(tree*& tr, tree* x) // Родитель красный, дядя, дед, брат и дети черные
{
	tree* G = grandparent(x); // Дед
	x->parent->color = true; // Родитель становится черным
	G->color = false; // Дед становится красным
	if ((x == x->parent->left) && (x->parent = G->left)) // Если Х - левый ребенок и родитель - левый ребенок, то правый поворот, иначе левый
		right_rotate(tr, G);
	else
		left_rotate(tr, G);
}

void insert_case4(tree*& tr, tree* x) // родитель красный, дядя черный
{
	tree* G = grandparent(x); // Дед
	if ((x == x->parent->right) && (x->parent == G->left)) // Если Х - правый ребенок и родитель - левый ребенок
	{
		left_rotate(tr, x->parent); // Левый поворот
		x = x->left; // Переходим к левому ребенку
	}
	else
	{
		if ((x == x->parent->left) && (x->parent == G->right)) // Иначе правый поворот
		{
			right_rotate(tr, x->parent);
			x = x->right;
		}
	}
	insert_case5(tr, x);
}

void insert_case3(tree*& tr, tree* x) // если родитель, Дядя и X - красные
{
	tree* U = uncle(x); // дядя
	tree* G = grandparent(x); // дедушка
	if (U && U->color == false && x->parent->color == false)  // если дядя и родитель красные
	{
		x->parent->color = true; // родитель становится черным
		U->color = true; // дядя становится черным
		G->color = false; // дед - красным

		if (!G->parent) // если у деда нет родителя - становится черным
			G->color = true;
		else
		{
			if (grandparent(G)) // если у деда есть дед
				insert_case3(tr, G);
			else
				return;
		}
	}
	else
		insert_case4(tr, x);
}

void insert_case2(tree*& tr, tree* x) // если родитель х - красный, то переходим к 3 случаю
{
	if (x->parent->color == false)
		insert_case3(tr, x);
	else
		return;
}

void insert_case1(tree*& tr, tree* x) // если узел - корень
{
	if (!x->parent)
		x->color = true; // становится черным
	else
		insert_case2(tr, x);
}

void insert(tree*& tr, tree* prev, int x) // Prev - предыдущий узел
{
	if ((x < prev->inf) && (!prev->left)) // Если Х меньше предыдущего и у него нет левого ребенка
	{
		prev->left = node(prev, x); // Становится левым ребенком prev
		insert_case1(tr, prev->left);
	}
	else
	{
		if ((x > prev->inf) && (!prev->right)) // Если Х больше предыдущего и у него нет правого ребенка
		{
			prev->right = node(prev, x); // Становится правым ребенком prev
			insert_case1(tr, prev->right);
		}
		else
		{
			if ((x < prev->inf) && (prev->left)) // Если меньше(больше) и есть левый(правый) ребенок, идем дальше по дереву
				insert(tr, prev->left, x);
			else
			{
				if ((x > prev->inf) && (prev->right))
				{
					insert(tr, prev->right, x);
				}
			}
		}
	}
}

// удаление
void del_element6(tree*& tr, tree* x) // брат, близкий племянник - черные, дальний племянник - красный, а его дети - черные
{
	tree* B = brother(x);

	B->color = x->parent->color; // брат становится того же цвета, что и родитель
	x->parent->color = true; // родитель становится черным

	if (x == x->parent->left)  // если Х - левый ребенок
	{
		B->right->color = true; // правый племянник становится черным
		left_rotate(tr, x->parent); // левый поворот
	}
	else
	{
		B->left->color = true; // левый племянник становится черным
		right_rotate(tr, x->parent); // правый поворот
	}
}

void del_element5(tree*& tr, tree* x) // брат, дальний племянник - черные, близкий племянник - красный, а его дети - черные
{
	tree* B = brother(x); // брат

	bool col_l; // цвет левого племянника
	bool col_r; // цвет правого племянника

	if (!B->left)
		col_l = true;
	else
		col_l = B->left->color;

	if (!B->right)
		col_r = true;
	else
		col_r = B->right->color;

	if (x == x->parent->left && col_l == false && col_r == true) // если Х - левый ребенок, левый племянник - красный, а правый - черный
	{
		B->color = false; // брат становится красным
		B->left->color = true; // левый племянник становится черным
		right_rotate(tr, B); // правый поврот
	}
	else
	{
		if (x == x->parent->right && col_l == true && col_r == false) // Если Х - правый ребенок, левый племянник - черный, а правый - красный
		{
			B->color = false; // брат становится красным
			B->right->color = true; // правый племянник становится черным
			left_rotate(tr, B); // левый поворот
		}
	}
	del_element6(tr, x);
}

void del_element4(tree*& tr, tree* x) // Х, брат, племянники - черные, родитель красный
{
	tree* B = brother(x); // брат

	bool col_l; // цвет л племянника
	bool col_r; // цвет п племянника

	if (!B->left)
		col_l = true;
	else
		col_l = B->left->color;

	if (!B->right)
		col_r = true;
	else
		col_r = B->right->color;

	if (x->parent->color == false && B->color == true && col_l == true && col_r == true) // Если родитель - красный, а брат и племянники - черные
	{
		B->color = false; // брат становится красным
		x->parent->color = true; // родитель становится черным
	}
	else
		del_element5(tr, x);
}

void del_element3(tree*& tr, tree* x) // Х, родитель, брат, племянники - черные
{
	tree* B = brother(x); // брат

	bool col_l; // цвет л племянника
	bool col_r; // цвет п племянника

	if (!B->left)
		col_l = true;
	else
		col_l = B->left->color;

	if (!B->right)
		col_r = true;
	else
		col_r = B->right->color;

	if (x->parent->color == true && B->color == true && col_l == true && col_r == true) // Если родитель, брат и племянники - черные
	{
		B->color = false; // брат становится красным

		if (!x->parent) // если Х - корень
		{
			if (x->left)
				x->left->color = true; // если есть левый ребенок - он черный
			else
			{
				tr = x;
				x->right->color = true;
			}
		}
		else
		{
			tree* B = brother(x);
			if (B->color == false)
			{
				x->parent->color = false;
				B->color = true;
				if (x == x->parent->left)
					left_rotate(tr, x->parent);
				else
					right_rotate(tr, x->parent);
			}
			del_element3(tr, x);
		}
	}
	else
		del_element4(tr, x);
}

void del_element2(tree*& tr, tree* x) // х - черный, родитель черный, брат красный
{
	tree* B = brother(x);
	if (B->color == false)
	{
		x->parent->color = false;
		B->color = true;
		if (x == x->parent->left)
			left_rotate(tr, x->parent);
		else
			right_rotate(tr, x->parent);
	}
	del_element3(tr, x);
}

void del_element1(tree*& tr, tree* x) // x - корень дерева, одна ветка
{
	if (!x->parent)
	{
		if (x->left)
			x->left->color = true;
		else {
			tr = x;
			x->right->color = true;
		}
	}
	else {
		del_element2(tr, x);
	}
}

void replace(tree*& tr, tree* x) // случай когда удаляем x с одним ребенком
{
	if (x->left)
	{
		if (x->parent)
		{
			tree* ch = x->left;
			ch->parent = x->parent;
			if (x == x->parent->left)
				x->parent->left = ch;
			else
				x->parent->right = ch;
		}
	}
	else
	{
		if (x->parent)
		{
			tree* ch = x->right;
			ch->parent = x->parent;
			if (x == x->parent->left)
				x->parent->left = ch;
			else
				x->parent->right = ch;
		}
	}
}

tree* Max_par(tree* tr)
{
	if (!tr->right)
		return tr; // нет правого ребенка
	else
		return Max_par(tr->right); // идем по правой ветке до конца
}

void delete_one(tree*& tr, tree* x) // удаление
{
	tree* buf = NULL;

	if (x->right && x->left) // есть два ребенка
	{

		buf = Max_par(x->left); // предыдущий по значению элемент

		int tpm = x->inf;
		x->inf = buf->inf; // меняем местами с предыдущим по значению элементом
		buf->inf = tpm;

		x = buf;
	}
	if (x->right || x->left) // есть один ребенок
	{
		tree* ch = NULL;
		if (x->left && !x->right)
			ch = x->left;
		if (x->right && !x->left)
			ch = x->right;

		replace(tr, x);
		if (x->color == true) // если удаляемый элемент был красным то тот, что на его месте перекрашиваем в красный (если нужно)
		{
			if (ch->color == false)
				ch->color = true;
			else
				del_element1(tr, x);
		}
	}
	else // нет детей 
	{
		if (!x->right && !x->left)
		{
			if (x->color == true) {
				del_element1(tr, x);
			}
			else
			{
				if (x == x->parent->left)
					x->parent->left = NULL;
				else
					x->parent->right = NULL;
			}
		}
	}

	if (!x->left && !x->right)
	{
		if (x == x->parent->left)
			x->parent->left = NULL;
		else
		{
			if (x == x->parent->right)
				x->parent->right = NULL;
		}
	}

	delete x;
}

tree* find(tree* tr, int x) // поиск
{
	if (!tr || x == tr->inf) // нашли или дошли до конца ветки
		return tr;
	if (x < tr->inf)
		return find(tr->left, x); // ищем по левой ветке
	else
		return find(tr->right, x); // ищем по правой ветке
}

int main() {
	setlocale(LC_ALL, "Russian");

	tree* tr = NULL;

	cout << "Введите количество элементов: \n";
	int n; cin >> n;

	cout << "Введите элементы: \n";
	int x; cin >> x;
	tr = root(x);
	for (int i = 0; i < n - 1; ++i) {
		cin >> x;
		insert(tr, tr, x);
	}

	cout << "Прямой обход: ";
	preorder_color(tr);
	cout << endl;

	for (int i = 0; i < 1; i++) {
		cout << "Введите элемент для удаления: ";
		cin >> x;
		delete_one(tr, find(tr, x));
		cout << endl;

		cout << "Прямой обход: ";
		preorder_color(tr);
		cout << endl;
	}


	for (int i = 0; i < 1; i++) {
		cout << "Введите элемент для удаления: ";
		cin >> x;
		delete_one(tr, find(tr, x));
		cout << endl;

		cout << "Прямой обход: ";
		preorder_color(tr);
		cout << endl;
	}



	// Ввод нового элемента
	cout << "Введите элемент для вставки: ";
	cin >> x;
	insert(tr, tr, x); // Вставляем введенный элемент

	// Вывод дерева после вставки
	cout << "Прямой обход после вставки нового элемента: ";
	preorder_color(tr);
	cout << endl;

	return 0;
}
\end{minted}

\section{Поиск в красно"=черном дереве}
Пусть у красно"=чёрного дерева будет высота h. Так как у красной вершины чёрные дети (по свойству 3), количество красных вершин не больше $h/2$. Тогда чёрных вершин не меньше, чем $h/2 - 1$. Для количества внутренних вершин в дереве выполняется неравенство $N \leq h/2 - 1$. Прологарифмировав неравенство, имеем: $\log{(N + 1)} \geq  h/2 \rightarrow 2\log{(N + 1)} \geq h \rightarrow h \leq 2\log{(N + 1)}$.

Во всех случаях операция поиска займёт не более $O(logN)$ времени, так как красно-чёрное дерево является сбалансированным.

\section{Вставка в красно-черном дереве}
\subsection{Лучший случай}
Никакие свойства не были нарушены либо произошло просто перекрашивание, занимающее константное время. Сложность — $O(logN)$.

\subsection{Худший случай}
Красно-чёрным деревьям требуется не более 2 поворотов для восстановления баланса, каждый из которых занимает константное время. Так что в худшем случае при вставке будет 2 оборота, и временная сложность составит $O(logN)$.

\subsection{Средний случай}
Средний случай является средним значением всех возможных случаев, следовательно, временная сложность вставки в этом случае также составит $O(logN)$.
Таким образом, временная сложность для всех случаев равна $O(logN)$.

\section{Удаление в красно-черном дереве}
\subsection{Лучший случай}
В лучшем случае вращений нет, и может происходить разве что перекрашивание за константное время, потому что мы просто обращаемся к областям памяти. Временная сложность составит $O(logN)$.

\subsection{Худший случай}
Красно-чёрным деревьям требуется не более 3 поворотов во время удаления. Так что в худшем случае при удалении будет 3 поворота и временная сложность составит $O(log N)$.

\subsection{Средний случай}
Средний случай является средним значением всех возможных случаев, следовательно, временная сложность вставки в этом случае также составит $O(log N)$.
Таким образом, временная сложность для всех случаев равна $O(logN)$.

\section{Обходы дерева}
Красно-чёрное дерево, также, как и дерево бинарного поиска имеет 3 основных обхода: прямой, обратный и симметричный. Их разница заключается в том, в каком порядке мы обращаемся к элементам. Каждый из них будет иметь временную сложность $O(N)$, так как процедура вызывается ровно два раза для каждого узла дерева.

\section{Расход памяти}
Расход памяти в красно-чёрном дереве происходит так же, как в двоичном дереве поиска, и определяется общим количеством узлов. Поэтому получаем O(N), потому что нам не нужно дополнительное пространство для хранения повторяющихся структур данных. Данный вывод следует из того, что каждый узел имеет три указателя: левый ребёнок, правый ребёнок и родитель. Каждый узел занимает O(1) памяти. Для отслеживания цвета каждого узла требуется только один бит информации на каждый узел. Во многих случаях дополнительный бит данных может храниться без дополнительных затрат памяти. В результате сложность по памяти будет равна O(N), где N – количество узлов в дереве.

Вращение и перекрашивание происходят за время $O(1)$. Временная сложность всех функций равна $O(logN)$, потому что дерево всегда сбалансированное.

% Отобразить все источники. Даже те, на которые нет ссылок.
% \nocite{*}

% \begin{thebibliography}{99}
	\bibliographystyle{ugost2003}

% \end{thebibliography}

% Окончание основного документа и начало приложений Каждая последующая секция
% документа будет являться приложением
\appendix

\end{document}
